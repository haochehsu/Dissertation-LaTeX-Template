\documentclass[letterpaper]{report}
\usepackage[colorlinks=true,linkcolor=Blue,urlcolor=Blue,citecolor=Blue,anchorcolor=Blue]{hyperref}
\usepackage{Style/arxiv, Style/package}

%%%%%%%%%% CUSTOM TOC %%%%%%%%%%
\usepackage{titletoc}
\titlecontents{chapter}
  [0em] % left margin
  {\vspace{1ex}}% Space above the entry
  {\normalfont\contentslabel{1.5em}\normalfont} % Number bold, title normal font
  {\hspace{-1.5em}}% Numbered entry format without number
  {\titlerule*[0.5pc]{.}\normalfont\contentspage} % Filler-page format (dot leaders and bold page number)
  
%%%%%%%%%% ABBREVIATIONS %%%%%%%%%%
\newcommand{\R}{\mathbb{R}}
\newcommand{\h}{\mathcal{H}}
\newcommand{\E}{\mathbb{E}}
\newcommand{\Pn}{\mathbb{P}_n}
\newcommand{\bs}{\boldsymbol}
\newcommand{\sgn}{\mathrm{sign}}
\newcommand{\diag}{\mathrm{diag}}
\newcommand{\F}{\mathcal{F}}
\newcommand{\trace}{\mathrm{trace}}
\newcommand{\cov}{\mathrm{cov}}
\newcommand{\var}{\mathrm{var}}
\newcommand{\A}{\mathcal{A}}
\newcommand{\B}{\mathcal{B}}
\newcommand{\x}{{\bs X}}
\newcommand{\g}{\cellcolor{gray!25}}
\newcommand{\gline}{\rowcolor{gray!25}}
\newtheorem{theorem}{Theorem}
\newtheorem*{theorem*}{Theorem}
\newtheorem*{definition}{Definition}
\newtheorem{assumption}{Assumption}
\newtheorem{assumptionalt}{Assumption}[assumption]
\newenvironment{assumptionp}[1]{
  \renewcommand\theassumptionalt{#1}
  \assumptionalt
}{\endassumptionalt}
\newtheorem{lemma}{Lemma}
\newtheorem{example}{Example}
\newtheorem{corollary}{Corollary}
\newtheorem{remark}{Remark}
\newtheorem{proposition}{Proposition}
\renewcommand{\headeright}{~}
\renewcommand{\undertitle}{~}
\renewcommand{\shorttitle}{~}
\def\lquote{\textquotedblleft}
\def\rquote{\textquotedblright \ }

%%%%%%%%%% STYLE %%%%%%%%%%
\def\target_page{iii}
\fancypagestyle{copyright}{\fancyhf{}
\renewcommand{\headrulewidth}{0pt}}

\fancypagestyle{preface}{\fancyhf{}
\renewcommand{\headrulewidth}{0pt}
\pagenumbering{roman}
\fancyfoot[C]{\hyperlink{page.\target_page}{{\color{black}\thepage}}}}

\fancypagestyle{text}{\fancyhf{}
\renewcommand{\headrulewidth}{0pt}
\pagenumbering{arabic}
\fancyfoot[C]{\hyperlink{page.\target_page}{{\color{black}\thepage}}}}

\fancypagestyle{plain}{\fancyhf{}
\fancyfoot[C]{\hyperlink{page.\target_page}{{\color{black}\thepage}}}}

%%%%%%%%%% CUSTOMIZE %%%%%%%%%%
\renewcommand*\listfigurename{LIST OF FIGURES}
\renewcommand*\listtablename{LIST OF TABLES}
\renewcommand{\contentsname}{TABLE OF CONTENTS}
\renewcommand*\thefootnote{\textcolor{red}{\arabic{footnote}}}
\definecolor{mycolor}{HTML}{3D8A2B}
\definecolor{darkerGreen}{HTML}{278041}
\definecolor{darkerPurple}{HTML}{744CB4}
\definecolor{clay}{HTML}{CA7C20}

%%%%%%%%%% LISTING %%%%%%%%%%
\lstdefinestyle{python_code}{
language=Python, % python, matlab, r
basicstyle=\linespread{1}\footnotesize\ttfamily,
numbers=left,
numberstyle=\tiny,
frame=tb,
columns=fullflexible,
showstringspaces=false,
breaklines=true}

%%%%%%%%%% FONT SIZE (DEFAULT: 10pt) %%%%%%%%%%
\makeatletter
% \input{size11.clo} % 11pt
\input{size12.clo} % 12pt
\makeatother

%%%%%%%%%% COVER %%%%%%%%%%
\thesistitle{Title of the Thesis}
\documenttitle{Dissertation}
\degreename{Doctor of Philosophy}
\degreefield{Economics}
\authorname{Your Name}
\committeechair{Professor A}
\othercommitteemembers{
Associate Professor B\\
Associate Professor C\\
Assistant Professor D}
\degreeyear{\the\year{}}

%%%%%%%%%% BEGIN DOCUMENT %%%%%%%%%%
\begin{document}
\thesistitlepage

%%%%%%%%%% COPYRIGHT %%%%%%%%%%
\pagestyle{copyright}
\vspace*{\fill}
\begin{center}
  Chapter 1 is reprinted from the Journal of JournalName , Vol. 527 (2021). Your Name, ``The Title of the Paper,'' © 2021 The Publisher, with permission as stated at https://URL.
  
  Portions of Chapter 2 \copyright \ 2020 Someone's Name
  
  All other materials \copyright \ \the\year{} Your Name
\end{center}

%%%%%%%%%% DEDICATION %%%%%%%%%%
\clearpage
\pagestyle{preface}
\setcounter{page}{2}

\begin{center}
  \section*{DEDICATION}
  To someone and someone for their support.
\end{center}

%%%%%%%%%% TOC %%%%%%%%%%
\clearpage
\makeatletter
\begin{center}
\section*{TABLE OF CONTENTS}
\end{center}
\vspace*{2.25em \@plus\p@}
\phantomsection
{\hypersetup{linkcolor=black}\@starttoc{toc}}
\makeatother

\clearpage
\makeatletter
\begin{center}
\section*{LIST OF FIGURES}
\end{center}
\vspace*{2.25em \@plus\p@}
\phantomsection
{\hypersetup{linkcolor=black}\@starttoc{lof}}
\makeatother
\addcontentsline{toc}{chapter}{\listfigurename}

\clearpage
\makeatletter
\begin{center}
\section*{LIST OF TABLES}
\end{center}
\vspace*{2.25em \@plus\p@}
\phantomsection
{\hypersetup{linkcolor=black}\@starttoc{lot}}
\makeatother
\addcontentsline{toc}{chapter}{\listtablename}

% \clearpage
% {\hypersetup{linkcolor=black}
% \tableofcontents}
% {\hypersetup{linkcolor=black}
% \listoffigures}
% \addcontentsline{toc}{chapter}{\listfigurename}
% {\hypersetup{linkcolor=black}
% \listoftables}
% \addcontentsline{toc}{chapter}{\listtablename}

%%%%%%%%%% ACKNOWLEDGMENTS %%%%%%%%%%
\clearpage
\onehalfspacing
\begin{center}
  \section*{ACKNOWLEDGMENTS}
  \addcontentsline{toc}{chapter}{ACKNOWLEDGMENTS}
\end{center}

I would like to thank NameOfCommitteeChair for his support and guidance over the past few years. \lipsum[69]

You must acknowledge grants and other funding assistance. I am grateful to some for her guidance. \lipsum[71]

I am incredibly thankful to my colleagues at UC Irvine for their friendship and encouragement. \lipsum[72]

Download this template at the \href{https://github.com/howardhsumail/Dissertation-LaTeX-Template.git}{Github repository}. \lipsum[75]

%%%%%%%%%% VITA %%%%%%%%%%
\clearpage
\onehalfspacing
\begin{center}
  \section*{VITA}
  \addcontentsline{toc}{chapter}{VITA}
  \textbf{Your Name}
\end{center}
\vspace*{2.25em}
{\parindent0pt
EDUCATION

Doctor of Philosophy in Economics\hfill YYYY\\[-0.1cm]
University of California, Irvine\hfill \textit{Irvine, CA}

Masters of Science in Economics\hfill YYYY\\[-0.1cm]
University of Wisconsin, Madison\hfill Madison, WI

Bachelor of Arts in Economics\hfill YYYY\\[-0.1cm]
University of California, Los Angeles\hfill Los Angeles, CA\\

RESEARCH EXPERIENCE

Graduate Student Researcher\hfill YYYY\\[-0.1cm]
University of California, Irvine\hfill Irvine, CA\\

TEACHING EXPERIENCE

Teaching Assistant\hfill YYYY-YYYY\\[-0.1cm]
University of California, Irvine\hfill Irvine, CA\\

FIELDS OF STUDY\\
Econometrics, Industrial Organization}

%%%%%%%%%% ABSTRACT %%%%%%%%%%
\clearpage
\doublespacing
\begin{center}
  \section*{ABSTRACT OF THE DISSERTATION}
  \addcontentsline{toc}{chapter}{ABSTRACT OF THE DISSERTATION}
  
  Title of the Thesis
  
  By
  
  Your Name
  
  Doctor of Philosophy in Economics
  
  University of California, Irvine, \the\year{}
  
  Professor NameOfCommitteeChair, Chair
\end{center}
\bigskip

The chapters of this dissertation explore different aspects in Economics. \lipsum[87-88]

%%%%%%%%%% TEXT %%%%%%%%%%
\newpage
\doublespacing
\pagestyle{text}
\setcounter{page}{1}

%%%%%%%%%% CH1 %%%%%%%%%%
\chapter{Title of Chapter One}
This chapter explores the aspect of firm competitions. \lipsum[11]\\

Download this template at the \href{https://github.com/howardhsumail/Dissertation-LaTeX-Template.git}{Github repository}.

\section{Introduction}

The {\color{mycolor}competition} can be illustrated with the following graph with the implementation is presented in Listing \ref{mypythoncode}:

\begin{figure}[H]
  \centering
  \caption{This is a graph}
  \hlabel{picture1}
  \includegraphics[scale=0.5]{Graph/pic.pdf}
  \hspace*{-0.6cm}
  \begin{minipage}{0.9\textwidth}
    \onehalfspacing
    \vspace*{0.12cm}
    \begin{tablenotes}
      \footnotesize
      \item\textit{Note:} some notes. The graph should be self-contained. \lipsum[66]
    \end{tablenotes}
  \end{minipage}
\end{figure}

\section{Model}
\lipsum[4] The proof is discussed in Appendix \ref{ch1_proof}.

\begin{theorem}[Envelope Theorem]
  Only the direct effects of a change in an exogenous variable need be considered, even though the exogenous variable may enter the maximum value function indirectly as part of the solution to the endogenous choice variables.
\end{theorem}

\section{Comparative Statics}
This is also demonstrated in Figure \ref{picture1}.

\begin{lstlisting}[style=python_code, caption={Long short-term memory}, label=mypythoncode]
class network_LSTM(nn.Module):
    def __init__(self, input_size=1, hidden_size=256, output_size=1):
        super().__init__()
        self.hidden_size = hidden_size
        self.lstm = nn.LSTM(input_size, hidden_size)

        # fully-connected
        self.linear = nn.Linear(hidden_size, output_size)

        self.hidden = (
            torch.zeros(1, 1, self.hidden_size),
            torch.zeros(1, 1, self.hidden_size)
        )

    def forward(self,vec):
        lstm_output, self.hidden = self.lstm(vec.view(len(vec),1,-1), self.hidden)
        prediction = self.linear(lstm_output.view(len(vec),-1))
        return prediction[-1]
\end{lstlisting}

\section{Conclusion}
\lipsum[7]

%%%%%%%%%% CH2 %%%%%%%%%%
\chapter{Title of Chapter Two}
This chapter explores two-side markets. \lipsum[30]

\section{Introduction}
We follow the approach from \cite{HL2019}. \lipsum[50] By using this approach, comparable results can be obtained \citep{CES2013}. \lipsum[52]

\section{Model}
\lipsum[53-54]

\section{Identification}
\lipsum[16] To calculate the ELBO\footnote{More information about the evidence lower bound (ELBO) can be found on the \href{https://en.wikipedia.org/wiki/Evidence_lower_bound}{Wikipedia}. }, we start from using the property of the KL-divergence.


\section{Empirical Results}
The results are presented in Appendix \ref{summary_b}. Download this template at the \href{https://github.com/howardhsumail/Dissertation-LaTeX-Template.git}{Github repository}.

\section{Application}
\lipsum[61]

\section{Conclusion}
\lipsum[70]

%%%%%%%%%% CH3 %%%%%%%%%%
\chapter{Title of Chapter Three}
This chapter estimates the treatment effects. \lipsum[75] The simplex is depicted in Appendix \ref{triangle}.

\section{Introduction}
Many previous research has has studied this problem \citep{Lee2018, DS2018}. Download this template at the \href{https://github.com/howardhsumail/Dissertation-LaTeX-Template.git}{Github repository}. \lipsum[79]

\section{Algorithm}
\lipsum[80-81]

\begin{algorithm}
  \SetKwInOut{Input}{Input}
  \SetKwInOut{Output}{Output}

  \underline{function Euclid} $(a,b)$\;
  \Input{Two nonnegative integers $a$ and $b$}
  \Output{$\gcd(a,b)$}
  \eIf{$b=0$}
  {
  return $a$\;
  }
  {
  return Euclid$(b,a\mod b)$\;
  }
  \caption{Euclid's algorithm for finding the greatest common divisor of two nonnegative integers}
\end{algorithm}

\section{Results}
\lipsum[67]

\renewcommand*\arraystretch{0.55}
\renewcommand{\tabcolsep}{25pt}
\begin{table}[H]
  \centering
  \caption{Summary Statistics}
  \hlabel{ss}
  \fontsize{10}{11}\selectfont
  \begin{tabular}{
    l
    *{5}{S[table-format=2.1]}
    }
    \toprule
    & \multicolumn{3}{c}{\bfseries Cohort} \\
    \cmidrule(l){2-4}
    & {2006} & {2007} & {2008} \\
    \midrule
    \bfseries Students registered & {1535} & {1584} & {1767}\\
    \addlinespace
    \bfseries Gender (\%) \\
    Male                         & 61.1 & 64.5 & 57.7\\
    Female                       & 38.9 & 35.5 & 42.3\\
    \addlinespace
    \bfseries Race (\%) \\
    White                        & 43.3 & 43.4 & 40.6\\
    Black                        & 29.8 & 33.4 & 34.8\\
    \bottomrule
  \end{tabular}
  \begin{minipage}{0.8\textwidth}
    \onehalfspacing
    \vspace*{0.05cm}
    \begin{tablenotes}
      \footnotesize
      \item\textit{Note:} Source: UCT Institutional Planning Department.
    \end{tablenotes}
  \end{minipage}
\end{table}

\lipsum[37]

\section{Conclusion}
\lipsum[103]

%%%%%%%%%% Bibliography %%%%%%%%%%
\bibliographystyle{Style/aea} % Style/ecta
\onehalfspacing
\bibliography{reference}
\addcontentsline{toc}{chapter}{Bibliography}

%%%%%%%%%% Appendices %%%%%%%%%%
\begin{appendices}
  \doublespacing
  
  %%%%%%%%%% Appendix A %%%%%%%%%%
  \chapter{Supplementary material for  Chapter 1}
  
  \section{Proof of Theorem}
  
  \begin{proof} \hlabel{ch1_proof}
    
    Given $y$, $x$, $\Delta$, $\nu$, $\eta$, $\mathcal{L}$=
    $\begin{pmatrix}
    1 & 2 & 3 & 4 & 5 \\
    3 & 4 & 5 & 6 & 7
    \end{pmatrix}$,
    and $\prod=\begin{vmatrix}
    A &B  &C \\
    D&  E& F
    \end{vmatrix}$, if
    
    \begin{center}
      \begin{tabular}{ll}
        $\begin{cases}
        \text{trade}, & p(\text{trade})=\dfrac{y}{v}\\
        \text{no trade}, & p(\text{no trade})=1-\dfrac{y}{v}
        \end{cases}$
      \end{tabular}
    \end{center}
    
    \vspace*{-1.2cm}
    
    \begin{align}\label{myEquation}
      \nonumber y&=\underset{\pi}{\E}\Big(\beta x + \epsilon\Big) + \dfrac{1}{2}\text{ (large frac)} + \tfrac{1}{2} \text{ (inline frac)}\\
      \nonumber &\neq\sum\limits_{i}\beta_i(\underbrace{\alpha+\xi}_{\text{variables}}) + \epsilon\\
      &\Longrightarrow \int_{0}^{10}r \left( \dfrac{r}{50} \right)dr\xlongequal{\text{text here}}\dfrac{r^{3}}{150}\biggr\rvert^{10}_{0}, \forall x\in (a,b)
    \end{align}
    So from $\widehat{ABCD}$, $\widetilde{ABCD}$, $\widehat{ABCD}$, $\overrightarrow{ABCD}$, and $\overline{ABCD}$, we get the desire $\underline{\text{result}}$.
  \end{proof}
  
  \begin{framed}
  Consider $g(x)=f(x)-x$, since $f(x)$ and $x$ are continuous, then by IVT: $\exists c\in(a,b)$ s.t. $g(c)=0\implies \exists c\in(a,b)$ s.t. $f(c)-c=0\implies f(c)=c.$
\end{framed}
  
  %%%%%%%%%% Appendix B %%%%%%%%%%
  \chapter{Supplementary material for Chapter 2}
  
  \setcounter{table}{0}
  
  \section{Descriptive Statistics}
  
  The data can be summarized by the tables below:
  
  \renewcommand*\arraystretch{1.7}
  \renewcommand{\tabcolsep}{2.5pt}
  \begin{table}[H]
    \centering
    \renewcommand{\thetable}{B.\arabic{table}a}
    \caption{First Table}
    \label{summary_a}
    \fontsize{10.5}{10.5}\selectfont
    \hspace*{0cm}\begin{tabular}{lrrrrrrrrr}
    \toprule
    Category                   & Total & Shares (\%) & Female & Male  & Asian & Black/AA & His./Latino & White/Cau. & Zeros (\%) \\ \hline
    child care                 & 19.39 & 0.08   & 12.32  & 20.12 & 23.14 & 63.78    & 20.24       & 19.00      & 0.07  \\
    eating                     & 30.35 & 6.12   & 35.97  & 6.23 & 24.61 & 21.58    & 38.18       & 2.02      & 0.00  \\
    education                  & 9.91  & 0.04   & 9.94   & 90.54  & 9.69  & 7.99     & 10.64        & 10.14      & 0.90  \\
    entertainment (not TV)     & 26.05 & 0.10   & 29.19  & 26.60 & 33.36 & 26.13    & 4.43       & 25.15      & 0.45  \\ \bottomrule
  \end{tabular}
  \hspace*{-0.65cm}
  \begin{minipage}{1.05\textwidth}
    \onehalfspacing
    \vspace*{0.05cm}
    \begin{tablenotes}
      \footnotesize
      \item\textit{Note:} This is the first table.
    \end{tablenotes}
  \end{minipage}
\end{table}

\renewcommand*\arraystretch{1.7}
\renewcommand{\tabcolsep}{4.1pt}
\begin{table}[H]
  \centering
  \addtocounter{table}{-1}
  \renewcommand{\thetable}{B.\arabic{table}b}
  \caption{Second Table}
  \label{summary_b}
  \fontsize{10.5}{10.5}\selectfont
  \hspace*{0cm}\begin{tabular}{lrrrrrrrrr}
  \toprule
  Category                   & Total & Shares (\%) & Female & Male  & Asian & Black/AA & His./Latino & White/Cau. & Zeros (\%) \\ \hline
  child care                 & 19.39 & 0.08   & 39.32  & 40.12 & 23.14 & 18.78    & 20.24       & 19.00      & 0.07  \\
  personal care              & 13.92 & 0.06   & 24.00  & 23.14 & 16.12 & 1.76    & 15.15       & 13.66      & 0.00  \\
  sports/exercise            & 20.44 & 0.08   & 20.38  & 31.00 & 24.99 & 25.48    & 20.71       & 20.07      & 0.53  \\
  TV                         & 28.61 & 0.12   & 48.47  & 9.93 & 2.35 & 63.70    & 29.22       & 80.20      & 0.46  \\ \bottomrule
\end{tabular}
\hspace*{-0.65cm}
\begin{minipage}{1.05\textwidth}
  \onehalfspacing
  \vspace*{0.05cm}
  \begin{tablenotes}
    \footnotesize
    \item\textit{Note:} This is the second table.
  \end{tablenotes}
\end{minipage}
\end{table}

%%%%%%%%%% Appendix C %%%%%%%%%%
\chapter{Supplementary material for Chapter 3}

\section{More Discussions}

We graph with \texttt{tikz} in \LaTeX. \lipsum[103]

\section{Graphical Representations}

\begin{multicols}{2}
  \raggedcolumns
  
  \begin{tikzpicture}[scale=.65]
    \draw [<-] (0,7.8) node [left] {$x_2,y_2$} -- (0,0);
    \draw [->] (0,0) -- (8.9,0) node [below] {$x_1,y_1$};
    \node [left] at (0,6.45) {$P$};
    \node [below] at (5.05,0) {$P'$};
    \node [left] at (4.3,3.2) {$x^n$};
    \node [below] at (2.4,5.4) {$y^f$};
    \node [right] at (5.75,3.4) {$x^f$};
    \draw (5.8,.5) -- (2.65,6.7);
    \draw (.5,7.2) -- (7.7,1.7);
    \draw (4.45,5.3) to [out=-90, in=140] (5.75,3.2) to [out=-40, in=160] (7.7,2.2);
    \draw (4.15,4.3) to [out=-90, in=120] (4.55,3) to [out=-60, in=160] (7.4,1.1);
    \draw (0,6.45) to [out=0, in=115] (4.4,3.2) to [out=-65, in=90] (5.05,0);
  \end{tikzpicture}
  
  \columnbreak
  
  \begin{tikzpicture}[scale=0.98]\hlabel{triangle}
    \draw [thick] (0,0) node [left] {$\Pi_3$} -- (3,1.8) node [below] {0} -- (3,5.4) node [right] {$\Pi_2$};
    \draw [thick] (3,1.8) -- (6,0) node [right] {$\Pi_1$};
    \draw [thick] (.4,.26) -- (3,4.8);
    \draw [thick] (.4,.26) -- (5.55,.26) -- (3,4.8);
    \node [thick,below] at (1,.26) {$(0,0,1)$};
    \node [thick,right] at (3,4.9) {$(0,1,0)$};
    \node [thick,right] at (5.65,.45) {$(1,0,0)$};
  \end{tikzpicture}
\end{multicols}

\section{More \texttt{Tikz}}

\begin{figure}[H]
  \centering
  \caption{Caption above figure}
  \begin{tikzpicture}[xscale=1, yscale=1.5]
    \draw[line width=0.35mm, ->] (-0.1,0) -- (8.2,0) node[right] {$p$};
    
    \draw[line width=0.35mm, ->] (0,-0.1) -- (0,4.2) node[above] {$W_G(p)$};
    
    \draw (4,0) node[below ]{$\mu$};
    \draw (0,4) node[left]{$\mu$};
    
    \draw[line width=0.3mm, blue, name path =A] (0,0) plot[domain=0:8] (\x,{1/16*(8-\x)^2}) node[anchor=south] at (8,4) {$W_{\overline{\pi}}(p)$};
    \draw[line width=0.3mm,red, name path =B] (0,4) -- (4,0) -- (8,0) node[anchor=north] at (8,4) {$W_{\underline{\pi}}(p)$};
    
    \draw[line width=0.3mm, orange, name path =C](0,0)
    plot[domain=0:6.82746] (\x,{4-1.08586 *\x + 0.0732323*\x^2}) node[anchor=north] at (8,2.5){$W_G(p)$};
    
    \draw[line width=0.3mm, orange, name path =D](6.82746,0)
    -- (8,0);
    
    \draw (4.4,0) node[below]{$p_0$};
    
    \node[outer sep=0pt, circle, fill=darkerPurple, inner sep=1pt, minimum size=1.5mm] (P) at
    (4.4,0.64) {};
    
    \draw[line width=0.3mm, darkerGreen] (8,0) -- (5.37538,0) -- (P) -- (2.75076,1.7221575361) -- (1.37539,2.6241)--(0,4) node[anchor=south] at (8,2.5){$W_{G^\prime}(p)$};
    
    \node[outer sep=0pt, circle, fill=clay, inner sep=1pt, minimum size=1.5mm] (Q1) at
    (2.75076,0) {};
    
    \draw (Q1) node[below]{$\gamma_0$};
    
    \node[outer sep=0pt, circle, fill=darkerPurple, inner sep=1pt, minimum size=1.5mm] (Q2) at
    (2.75076,1.7221575361) {};
    
    \path[draw=red, dashed] (Q1) -- (Q2);
    \path[draw=red, dashed] (4.4,0) -- (P);
  \end{tikzpicture}
  \label{fig:2}
\end{figure}

\begin{figure}[H]
  \centering
  \begin{tikzpicture}[xscale=10,yscale=5,
    userDefineLine/.style={line width=0.4mm, red}
    ]
    \draw[line width=0.3mm, ->] (0,0) -- (1.04,0);
    \node (B) at (1.1,0) {\text{prior}};
    
    \draw[line width=0.3mm, ->] (0,0) -- (0,0.8);
    \node (B) at (0.32,0.32) {$V_1^*(p^0)$};
    \draw (0,0) node[left]{\footnotesize $0$};
    
    \draw[userDefineLine] (0,0) -- (0.3,0);
    \draw[userDefineLine] (0.3, 0) -- (1,0.7);
    
    \draw[line width=0.3mm, red, dotted] (0.5,0) -- (0.5, 0.35);
    
    \draw[line width=0.6mm, violet, opacity=0.7, dashed] (0,0) -- (1,0.7);
    
    % x axis
    \draw (0.3,0) coordinate(g1) node[below ]{$p$};
    \draw (g1) ++(0,-0.01) -- ++(0,0.02);
    
    \draw (1,0) coordinate(g2) node[below]{$1$};
    \draw (g2) ++(0,-0.01) -- ++(0,0.02);
    
    \draw[dotted] (1,0) -- (1,0.7);
    
    % y axis
    \draw (0,0.7) coordinate(u1) node[left]{\footnotesize $1-p$};
    \draw (u1) ++(-0.01,0) -- ++(0.02,0);
    
    \draw[dotted] (u1) -- (1,0.7);
    
    %Legend
    \matrix [draw, fill=white, below right] at (1.05,0.5) {
    \draw [userDefineLine] ++(-0.3,0) -- ++(0.6,0) node[black,right] {$V_1(p^0)$}; \\
    \draw [line width=0.6mm, violet, opacity=0.7, dashed] ++(-0.3,0) -- ++(0.6,0) node[black,right,opacity=1] {$V_1^*(p^0)$}; \\
    };
    
    \node[outer sep=0pt, circle, fill=blue, inner sep=1pt, minimum size=1.5mm] (P) at
    (0.5,0.35) {};
    
    \node (C) at (0.5,-0.06) {\color{blue}$p^0=\frac{1}{2}$};
    
  \end{tikzpicture}
  \caption{Caption below figure} \label{fig:M2}
\end{figure}

\begin{figure}[H]
  \centering
  \begin{tikzpicture}[scale=10.0, xscale=2]
    \draw[line width=0.3mm, ->] (1/2, 0) -- (1, 0) node[right] {$\lambda_i$};
    \draw[line width=0.3mm, ->] (1/2, 0) -- (1/2, 1/3) node[above] {$R_j$};
    
    % blue lines
    \draw[line width=0.3mm, scale=1, domain=1/2:11/16, smooth, variable=\x, blue] plot ({\x}, {1/5});
    \draw[line width=0.3mm, scale=1, domain=11/16:1, smooth, variable=\x, blue] plot ({\x}, {1/4-1/8-1/2*(1/8-((2*\x-1)/2)*(2/3)});
    
    % red lines
    \draw[line width=0.3mm, scale=1, domain=1/2:5/8, smooth, variable=\x, red] plot ({\x}, {1/4});
    \draw[line width=0.3mm, scale=1, domain=5/8:1, smooth, variable=\x, red] plot ({\x}, {1/4-1/8-1/2*(1/8-((2*\x-1)/2))});
    
    \draw[line width=0.3mm, dashed,black,thick] (5/8,1/4)--(1,1/4);
    \draw[line width=0.3mm, dashed,black,thick] (11/16,1/5)--(1,1/5);
    
    \node[circle,inner sep=1pt,fill=black,label=left:{$\frac{1}{4}$}] at (1/2,1/4) {};
    \node[circle,inner sep=1pt,fill=red,minimum size=2mm,label=below:{$\frac{5}{8}$}] at (5/8,0) {};
    \node[circle,inner sep=1pt,fill=blue,minimum size=2mm, label=below:{$
    \frac{11}{16}$}] at (11/16,0) {};
    \node[circle,inner sep=1pt,fill=black,label=below:{$\frac{1}{2}$}] at (1/2,0) {};
    
    \matrix [draw, above left] at (1.11,0.08) {
    &\node[red,font=\fontsize{10}{10}\selectfont] {$\mu=1$}; \\
    &\node[blue,font=\fontsize{10}{10}\selectfont] {$\mu=2/3$}; \\
    };
    
    \node (D) at (0.8,-0.04) {$p=\frac{1}{4}$};
  \end{tikzpicture}
  \caption{Caption below figure} \label{fig:M3}
\end{figure}

\end{appendices}

\end{document}
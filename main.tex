\documentclass[letterpaper]{report}
\usepackage{Style/arxiv, Style/package}

%%%%%%%%%% ABBREVIATION %%%%%%%%%%
\newcommand{\R}{\mathbb{R}}
\newcommand{\h}{\mathcal{H}}
\newcommand{\E}{\mathbb{E}}
\newcommand{\Pn}{\mathbb{P}_n}
\newcommand{\bs}{\boldsymbol}
\newcommand{\sgn}{\mathrm{sign}}
\newcommand{\diag}{\mathrm{diag}}
\newcommand{\F}{\mathcal{F}}
\newcommand{\trace}{\mathrm{trace}}
\newcommand{\cov}{\mathrm{cov}}
\newcommand{\var}{\mathrm{var}}
\newcommand{\A}{\mathcal{A}}
\newcommand{\B}{\mathcal{B}}
\newcommand{\x}{{\bs X}}
\newcommand{\g}{\cellcolor{gray!25}}
\newcommand{\gline}{\rowcolor{gray!25}}
\newtheorem{theorem}{Theorem}
\newtheorem*{theorem*}{Theorem}
\newtheorem*{definition}{Definition}
\newtheorem*{assumption}{Assumption}
\newtheorem{lemma}{Lemma}
\newtheorem{example}{Example}
\newtheorem{corollary}{Corollary}
\newtheorem{remark}{Remark}
\newtheorem{proposition}{Proposition}
\renewcommand{\headeright}{~}
\renewcommand{\undertitle}{~}
\renewcommand{\shorttitle}{~}
\def\lquote{\textquotedblleft}
\def\rquote{\textquotedblright \ }

%%%%%%%%%% STYLE %%%%%%%%%%
\fancypagestyle{cover}{\fancyhf{}
\renewcommand{\headrulewidth}{0pt}
\fancyfoot[C]{\the\year{}}}

\fancypagestyle{copyright}{\fancyhf{}
\renewcommand{\headrulewidth}{0pt}
\fancyfoot[C]{Portions of Chapter 2 \copyright \ 2020 Someone's Name\\All other materials \copyright \ \the\year{} Your Name}}

\fancypagestyle{preface}{\fancyhf{}
\renewcommand{\headrulewidth}{0pt}
\pagenumbering{roman}
\fancyfoot[C]{\thepage}}

\fancypagestyle{text}{\fancyhf{}
\renewcommand{\headrulewidth}{0pt}
\pagenumbering{arabic}
\def\tocpage{iii}
\cfoot{\hyperlink{page.\tocpage}{{\color{black}\thepage}}}}

%%%%%%%%%% CUSTOMIZE %%%%%%%%%%
\hypersetup{colorlinks=true,linkcolor=blue,urlcolor=blue,citecolor=blue,anchorcolor=blue}
\definecolor{mycolor}{HTML}{3D8A2B}
\renewcommand*\thefootnote{\textcolor{red}{\arabic{footnote}}}
\renewcommand*\listfigurename{LIST OF FIGURES}
\renewcommand*\listtablename{LIST OF TABLES}
\renewcommand{\contentsname}{TABLE OF CONTENTS}

%%%%%%%%%% LISTING %%%%%%%%%%
\lstdefinestyle{python_code}{
language=Python, % python, matlab, r
basicstyle=\linespread{1}\footnotesize\ttfamily,
numbers=left,
numberstyle=\tiny,
frame=tb,
columns=fullflexible,
showstringspaces=false,
breaklines=true
}

%%%%%%%%%% BEGIN DOCUMENT %%%%%%%%%%
\begin{document}

%%%%%%%%%% COVER %%%%%%%%%%
\pagestyle{cover}
{\fontsize{12.5}{12.5}\selectfont
\begin{center}
  
  \vspace*{1.72cm}
  UNIVERSITY OF CALIFORNIA,\\IRVINE
  
  \vspace*{1cm}
  Title of the Thesis
  
  \vspace*{0.25cm}
  DISSERTATION
  
  \vspace*{0.9cm}
  Submitted in Partial Satisfaction of the Requirements
  
  for the degree of
  
  \vspace*{0.73cm}
  DOCTOR OF PHILOSOPHY
  
  \vspace*{0.3cm}
  in Economics
  
  \vspace*{0.9cm}
  by
  
  \vspace*{0.9cm}
  Your Name
\end{center}

\vfill
\begin{flushright}
  Dissertation Committee:
  
  Professor Name1 (Chair)
  
  Associate Professor Name2
  
  Associate Professor Name3
  
  Assistant Professor Name4
\end{flushright}
}

%%%%%%%%%% COPYRIGHT %%%%%%%%%%
\newpage

\vfill
\
\pagestyle{copyright}

\newpage
\pagestyle{preface}
\setcounter{page}{2}

\begin{center}
  \section*{DEDICATION}
  
  To someone for their support.
\end{center}

%%%%%%%%%% TOC %%%%%%%%%%
\newpage
{\hypersetup{linkcolor=black}
\tableofcontents}
\newpage
{\hypersetup{linkcolor=black}
\listoffigures}
\addcontentsline{toc}{chapter}{\listfigurename}
{\hypersetup{linkcolor=black}
\listoftables}
\addcontentsline{toc}{chapter}{\listtablename}

%%%%%%%%%% ACKNOWLEDGMENTS %%%%%%%%%%
\doublespacing
\newpage
\begin{center}
  \section*{ACKNOWLEDGMENTS}
  \addcontentsline{toc}{chapter}{ACKNOWLEDGMENTS}
\end{center}

I would like to thank NameOfCommitteeChair for his support and guidance over the past few years.

I am grateful to some for her guidance.

I am incredibly thankful to my colleagues at UC Irvine for their friendship and encouragement.

I would like to thank many people. Last, but not least, I thank someone for their support over the years.

Download this template at the \href{https://github.com/howardhsumail/Dissertation-Template.git}{Github repository}.

%%%%%%%%%% CURRICULUM VITAE %%%%%%%%%%
\newpage
\begin{center}
  \section*{CURRICULUM VITAE}
  \addcontentsline{toc}{chapter}{CURRICULUM VITAE}
  
  {\fontsize{12}{12}\selectfont
  \textbf{Your Name}
  }
\end{center}

\textbf{EDUCATION}

\textbf{Doctor of Philosophy in Economics\hfill YYYY}\\[-0.2cm]
University of California, Irvine\hfill \textit{Irvine, CA}

\textbf{Masters of Science in Economics\hfill YYYY}\\[-0.2cm]
University of Wisconsin, Madison\hfill \textit{Madison, WI}

\textbf{Bachelor of Arts in Economics\hfill YYYY}\\[-0.2cm]
University of California, Los Angeles\hfill \textit{Los Angeles, CA}\\

\textbf{FIELDS OF STUDY}\\
Econometrics, Industrial Organization

%%%%%%%%%% ABSTRACT %%%%%%%%%%
\newpage
\begin{center}
  \section*{ABSTRACT OF THE DISSERTATION}
  \addcontentsline{toc}{chapter}{ABSTRACT OF THE DISSERTATION}
  
  Title of the Thesis
  
  By
  
  Your Name
  
  Doctor of Philosophy in Economics
  
  University of California, Irvine, \the\year{}
  
  Professor NameOfCommitteeChair, Chair
\end{center}
\vspace*{1cm}

The chapters of this dissertation explore different aspects in Economics.

%%%%%%%%%% TEXT %%%%%%%%%%
\newpage
\pagestyle{text}
\setcounter{page}{1}
\doublespacing

%%%%%%%%%% CH1 %%%%%%%%%%
\chapter{Title of Chapter One}
This chapter explores the aspect of firm competitions. \lipsum[11]\\

Download this template at the \href{https://github.com/howardhsumail/Dissertation-LaTeX-Template.git}{Github repository}.

\section{Introduction}

The {\color{mycolor}competition} can be illustrated with the following graph with the implementation is presented in Listing \ref{mypythoncode}:

\begin{figure}[H]
  \centering
  \caption{This is a graph}
  \hlabel{picture1}
  \includegraphics[scale=0.5]{Graph/pic.pdf}
  \hspace*{-0.6cm}
  \begin{minipage}{0.9\textwidth}
    \onehalfspacing
    \vspace*{0.12cm}
    \begin{tablenotes}
      \footnotesize
      \item\textit{Note:} some notes. The graph should be self-contained. \lipsum[66]
    \end{tablenotes}
  \end{minipage}
\end{figure}

\section{Model}
\lipsum[4] The proof is discussed in Appendix \ref{ch1_proof}.

\begin{theorem}[Envelope Theorem]
  Only the direct effects of a change in an exogenous variable need be considered, even though the exogenous variable may enter the maximum value function indirectly as part of the solution to the endogenous choice variables.
\end{theorem}

\section{Comparative Statics}
This is also demonstrated in Figure \ref{picture1}.

\begin{lstlisting}[style=python_code, caption={Long short-term memory}, label=mypythoncode]
class network_LSTM(nn.Module):
    def __init__(self, input_size=1, hidden_size=256, output_size=1):
        super().__init__()
        self.hidden_size = hidden_size
        self.lstm = nn.LSTM(input_size, hidden_size)

        # fully-connected
        self.linear = nn.Linear(hidden_size, output_size)

        self.hidden = (
            torch.zeros(1, 1, self.hidden_size),
            torch.zeros(1, 1, self.hidden_size)
        )

    def forward(self,vec):
        lstm_output, self.hidden = self.lstm(vec.view(len(vec),1,-1), self.hidden)
        prediction = self.linear(lstm_output.view(len(vec),-1))
        return prediction[-1]
\end{lstlisting}

\section{Conclusion}
\lipsum[7]

%%%%%%%%%% CH2 %%%%%%%%%%
\chapter{Title of Chapter Two}
This chapter explores two-side markets. \lipsum[30]

\section{Introduction}
We follow the approach from \cite{HL2019}. \lipsum[50] By using this approach, comparable results can be obtained \citep{CES2013}. \lipsum[52]

\section{Model}
\lipsum[53-54]

\section{Identification}
\lipsum[16] To calculate the ELBO\footnote{More information about the evidence lower bound (ELBO) can be found on the \href{https://en.wikipedia.org/wiki/Evidence_lower_bound}{Wikipedia}. }, we start from using the property of the KL-divergence.


\section{Empirical Results}
The results are presented in Appendix \ref{summary_b}. Download this template at the \href{https://github.com/howardhsumail/Dissertation-LaTeX-Template.git}{Github repository}.

\section{Application}
\lipsum[61]

\section{Conclusion}
\lipsum[70]

%%%%%%%%%% CH3 %%%%%%%%%%
\chapter{Title of Chapter Three}
This chapter estimates the treatment effects. \lipsum[75] The simplex is depicted in Appendix \ref{triangle}.

\section{Introduction}
Many previous research has has studied this problem \citep{Lee2018, DS2018}. Download this template at the \href{https://github.com/howardhsumail/Dissertation-LaTeX-Template.git}{Github repository}. \lipsum[79]

\section{Algorithm}
\lipsum[80-81]

\begin{algorithm}
  \SetKwInOut{Input}{Input}
  \SetKwInOut{Output}{Output}

  \underline{function Euclid} $(a,b)$\;
  \Input{Two nonnegative integers $a$ and $b$}
  \Output{$\gcd(a,b)$}
  \eIf{$b=0$}
  {
  return $a$\;
  }
  {
  return Euclid$(b,a\mod b)$\;
  }
  \caption{Euclid's algorithm for finding the greatest common divisor of two nonnegative integers}
\end{algorithm}

\section{Results}
\lipsum[67]

\renewcommand*\arraystretch{0.55}
\renewcommand{\tabcolsep}{25pt}
\begin{table}[H]
  \centering
  \caption{Summary Statistics}
  \hlabel{ss}
  \fontsize{10}{11}\selectfont
  \begin{tabular}{
    l
    *{5}{S[table-format=2.1]}
    }
    \toprule
    & \multicolumn{3}{c}{\bfseries Cohort} \\
    \cmidrule(l){2-4}
    & {2006} & {2007} & {2008} \\
    \midrule
    \bfseries Students registered & {1535} & {1584} & {1767}\\
    \addlinespace
    \bfseries Gender (\%) \\
    Male                         & 61.1 & 64.5 & 57.7\\
    Female                       & 38.9 & 35.5 & 42.3\\
    \addlinespace
    \bfseries Race (\%) \\
    White                        & 43.3 & 43.4 & 40.6\\
    Black                        & 29.8 & 33.4 & 34.8\\
    \bottomrule
  \end{tabular}
  \begin{minipage}{0.8\textwidth}
    \onehalfspacing
    \vspace*{0.05cm}
    \begin{tablenotes}
      \footnotesize
      \item\textit{Note:} Source: UCT Institutional Planning Department.
    \end{tablenotes}
  \end{minipage}
\end{table}

\lipsum[37]

\section{Conclusion}
\lipsum[103]

%%%%%%%%%% Bibliography %%%%%%%%%%
\bibliographystyle{Style/aea} % Style/ecta
\bibliography{reference}
\addcontentsline{toc}{chapter}{Bibliography}

%%%%%%%%%% Appendices %%%%%%%%%%
\begin{appendices}
  
  %%%%%%%%%% Appendix A %%%%%%%%%%
  \chapter{}
  \hlabel{ch1_proof}We will proof the following equation:
  \begin{proof}
    
    Given $y$, $x$, $\Delta$, $\nu$, $\eta$, $\mathcal{L}$=
    $\begin{pmatrix}
    1 & 2 & 3 & 4 & 5 \\
    3 & 4 & 5 & 6 & 7
  \end{pmatrix}$, and $\prod=\begin{vmatrix}
  A &B  &C \\
  D&  E& F
  \end{vmatrix}$, if
  
  \begin{center}
    \begin{tabular}{ll}
      $\begin{cases}
      \text{trade}, & p(\text{trade})=\dfrac{y}{v}\\
      \text{no trade}, & p(\text{no trade})=1-\dfrac{y}{v}
      \end{cases}$\\
    \end{tabular}
  \end{center}
  
  then we have
  
  \begin{align}
    \nonumber y&=\underset{\pi}{\E}\Big(\beta x + \epsilon\Big)\\
    &\neq\sum\limits_{i}\beta_i(\underbrace{\alpha+\xi}_{\text{variables}}) + \epsilon\\
    &\Longrightarrow \int_{0}^{10}r \left( \dfrac{r}{50} \right)dr\xlongequal{\text{text here}}\dfrac{r^{3}}{150}\biggr\rvert^{10}_{0}, \forall x\in (a,b)
  \end{align}
  So from $\widehat{ABCD}$, $\widetilde{ABCD}$, $\widehat{ABCD}$, $\overrightarrow{ABCD}$, and $\overline{ABCD}$, we get the desire $\underline{\text{result}}$.
\end{proof}

\begin{framed}
  Consider $g(x)=f(x)-x$, since $f(x)$ and $x$ are continuous, then $g:[a,b]\to\mathbb{R}$ is continuous. Then
  $$g(a)=f(a)-a>0, \ g(b)=f(b)-b<0$$
  By IVT: $\exists c\in(a,b)$ s.t. $g(c)=0\implies \exists c\in(a,b)$ s.t. $f(c)-c=0\implies f(c)=c.$
\end{framed}

%%%%%%%%%% Appendix B %%%%%%%%%%
\chapter{}\setcounter{table}{0}

The data can be summarized by the tables below:

\renewcommand*\arraystretch{0.95}
\renewcommand{\tabcolsep}{6pt}
\begin{table}[H]
  \renewcommand{\thetable}{B.\arabic{table}a}
  \caption{First Table}
  \label{summary_a}
  \fontsize{9}{11}\selectfont
  \hspace*{-0.5cm}
  \begin{tabular}{lrrrrrrrrr}
    \toprule
    Category                   & Total & Shares (\%) & Female & Male  & Asian & Black/AA & His./Latino & White/Cau. & Zeros (\%) \\ \hline
    child care                 & 19.39 & 0.08   & 12.32  & 20.12 & 23.14 & 63.78    & 20.24       & 19.00      & 0.07  \\
    eating                     & 30.35 & 6.12   & 35.97  & 6.23 & 24.61 & 21.58    & 38.18       & 2.02      & 0.00  \\
    education                  & 9.91  & 0.04   & 9.94   & 90.54  & 9.69  & 7.99     & 10.64        & 10.14      & 0.90  \\
    entertainment (not TV)     & 26.05 & 0.10   & 29.19  & 26.60 & 33.36 & 26.13    & 4.43       & 25.15      & 0.45  \\ \bottomrule
  \end{tabular}
  \hspace*{-1cm}
  \begin{minipage}{1.065\textwidth}
    \onehalfspacing
    \vspace*{0.05cm}
    \begin{tablenotes}
      \footnotesize
      \item\textit{Note:} This is the first table.
    \end{tablenotes}
  \end{minipage}
\end{table}

\renewcommand*\arraystretch{0.95}
\renewcommand{\tabcolsep}{7.5pt}
\begin{table}[H]
  \addtocounter{table}{-1}
  \renewcommand{\thetable}{B.\arabic{table}b}
  \caption{Second Table}
  \label{summary_b}
  \fontsize{9}{11}\selectfont
  \hspace*{-0.5cm}
  \begin{tabular}{lrrrrrrrrr}
    \toprule
    Category                   & Total & Shares (\%) & Female & Male  & Asian & Black/AA & His./Latino & White/Cau. & Zeros (\%) \\ \hline
    child care                 & 19.39 & 0.08   & 39.32  & 40.12 & 23.14 & 18.78    & 20.24       & 19.00      & 0.07  \\
    personal care              & 13.92 & 0.06   & 24.00  & 23.14 & 16.12 & 1.76    & 15.15       & 13.66      & 0.00  \\
    sports/exercise            & 20.44 & 0.08   & 20.38  & 31.00 & 24.99 & 25.48    & 20.71       & 20.07      & 0.53  \\
    TV                         & 28.61 & 0.12   & 48.47  & 9.93 & 2.35 & 63.70    & 29.22       & 80.20      & 0.46  \\ \bottomrule
  \end{tabular}
  \hspace*{-1cm}
  \begin{minipage}{1.065\textwidth}
    \onehalfspacing
    \vspace*{0.05cm}
    \begin{tablenotes}
      \footnotesize
      \item\textit{Note:} This is the second table.
    \end{tablenotes}
  \end{minipage}
\end{table}

\begin{multicols}{2}
  \raggedcolumns
  
  \begin{tikzpicture}[scale=.65]
    \draw [<-] (0,7.8) node [left] {$x_2,y_2$} -- (0,0);
    \draw [->] (0,0) -- (8.9,0) node [below] {$x_1,y_1$};
    \node [left] at (0,6.45) {$P$};
    \node [below] at (5.05,0) {$P'$};
    \node [left] at (4.3,3.2) {$x^n$};
    \node [below] at (2.4,5.4) {$y^f$};
    \node [right] at (5.75,3.4) {$x^f$};
    \draw (5.8,.5) -- (2.65,6.7);
    \draw (.5,7.2) -- (7.7,1.7);
    \draw (4.45,5.3) to [out=-90, in=140] (5.75,3.2) to [out=-40, in=160] (7.7,2.2);
    \draw (4.15,4.3) to [out=-90, in=120] (4.55,3) to [out=-60, in=160] (7.4,1.1);
    \draw (0,6.45) to [out=0, in=115] (4.4,3.2) to [out=-65, in=90] (5.05,0);
  \end{tikzpicture}
  
  \columnbreak
  
  \begin{tikzpicture}[scale=0.9]
    \draw [thick] (0,0) node [left] {$\Pi_3$} -- (3,1.8) node [below] {0} -- (3,5.4) node [right] {$\Pi_2$};
    \draw [thick] (3,1.8) -- (6,0) node [right] {$\Pi_1$};
    \draw [thick] (.4,.26) -- (3,4.8);
    \draw [thick] (.4,.26) -- (5.55,.26) -- (3,4.8);
    \node [thick,below] at (1,.26) {$(0,0,1)$};
    \node [thick,right] at (3,5) {$(0,1,0)$};
    \node [thick,right] at (5.65,.4) {$(1,0,0)$};
  \end{tikzpicture}
  
\end{multicols}
\end{appendices}

\end{document}